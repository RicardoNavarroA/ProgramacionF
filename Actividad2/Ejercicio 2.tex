\documentclass{article}

% set font encoding for PDFLaTeX or XeLaTeX
\usepackage{ifxetex}
\ifxetex
  \usepackage{fontspec}
\else
  \usepackage[T1]{fontenc}
  \usepackage[utf8]{inputenc}
  \usepackage{lmodern}
\fi

% used in maketitle
\title{Actividad 2}
\author{Ricardo Navarro Alvarado \\
 Departamento de Fisica \\
 Universidad de Sonora}
\date{11 de Septiembre del 2017}

% Enable SageTeX to run SageMath code right inside this LaTeX file.
% documentation: http://mirrors.ctan.org/macros/latex/contrib/sagetex/sagetexpackage.pdf
% \usepackage{sagetex}

\begin{document}
\maketitle
%comentarios

\section{Introduccion}
Tiro parabolico
\subsection{Archivo projectile}
Utilize esté archívo para obtener datos generales sobre el tiro parabólico
\begin{verbatim}
program projectile
  implicit none

  ! definimos constantes
  real, parameter :: g = 9.8
  real, parameter :: pi = 3.1415927

  ! definimos las variables
  real :: a, t, u, x, y
  real :: theta, v, vx, vy

  ! Leer valores para el ángulo a, el tiempo t, y la velocidad inicial u desde la terminal
  write(*,*) 'Dame el ángulo, el tiempo y la rapidez inicial'
  read(*,*) a, t, u

  ! convirtiendo ángulo a radianes
  a = a * pi / 180.0
  
  ! las ecuaciones de la posición en x y y
  x = u * cos(a) * t
  y = u * sin(a) * t - 0.5 * g * t * t

  ! La velocidad al tiempo t
  vx = u * cos(a)
  vy = u * sin(a) - g * t
  v = sqrt(vx * vx + vy * vy)
  theta = atan(vy / vx) * 180.0 / pi
 
 ! escribiendo el resultado en la pantalla
  write(*,*) 'x: ',x,'  y: ',y
  write(*,*) 'v: ',v,'  theta: ',theta

end program projectile
\end{verbatim}
\subsubsection{Ejemplo del programa}
Para $a$=$45$, $t$=$9$, y $u$=$5$ nos dio los datos:
$x$ = $31.819805$, $y$ = $-365.080231$, $v$ = $84.7382660$, y $\Theta$ = $-87.6087494$
\subsection{Archivo Tiempo de vuelo}
Este archivo se utilizo para obtener el tiempo de vuelo de la masa
\begin{verbatim}
program tiempo_vuelo
  implicit none

  ! definimos constantes
  real, parameter :: g = 9.8
  real, parameter :: pi = 3.1415927

  ! definimos las variables
  real :: a, t, u, x, y
  real :: theta, v, vx, vy

  ! Leer valores para el ángulo a, y la velocidad inicial u desde la terminal
  write(*,*) 'Dame el ángulo y la rapidez inicial'
  read(*,*) a, u

  !convirtiendo angulo a radianes
  a = a * pi / 180.0

  ! La ecuacion del tiempo de vuelo
  t = 2.0 * u * sin(a) / g

  ! escribiendo el resultado en la pantalla
  write(*,*) 't: ',t

end program tiempo_vuelo
\end{verbatim}
\subsubsection{Ejemplo del programa}
Para $a$=$45$ y $u$=$9$ nos dio un tiempo de vuelo $t$=$1.29876745$
\subsection{Archivo Altura maxima}
Utilize este archivo para conseguir la distancia vertical maxima
\begin{verbatim}
program altura_max
  implicit none

  ! definimos constantes
  real, parameter :: g = 9.8
  real, parameter :: pi = 3.1415927

  ! definimos variables
  real :: a, t, u, x, y, h
  real :: theta, v, vx, vy

  ! Leer valores para el ángulo a, el tiempo t, y la velocidad inicial u desde la terminal
  write(*,*) 'Dame el ángulo, y la velocidad inicial'
  read(*,*) a, u

  ! convirtiendo angulo a radianes
  a = a * pi / 180.0

  ! ecuacion de la velocidad final para la altura maxima
  v= u * sin(a) - g * t

  ! ecuacion del tiempo para la altura maxima
  t = u * sin(a) / g

  ! ecuaciones de la altura maxima
  h = u * t * sin(a) - 0.5 * g * t * t
  h = u * u * (sin(a)) * (sin(a)) / 2.0 * g

  ! escribiendo el resultado en la pantalla
  write(*,*) 't: ',t,' h: ',h

end program altura_max
\end{verbatim}
\subsubsection{Ejemplo del programa}
Para $a$=$45$, y $u$=$9$ nos dio un tiempo $t$=$0.649383724 $ y una altura $h$=$198.449997$
\subsection{Archivo Distancia maxima}
Este programa tiene el proposito de proporcionar la distancia maxima en $x$
\begin{verbatim}
program distancia_max
  implicit none

  ! definimos constantes
  real, parameter :: g = 9.8
  real, parameter :: pi = 3.1415927

  ! definimos las variables
  real :: a, t, u, x, y, d, h
  real :: theta, v, vx, vy

  ! Leer valores para el ángulo a, y la velocidad inicial u desde la terminal
  write(*,*) 'Dame el ángulo y la velocidad inicial'
  read(*,*) a, u

  ! convirtiendo angulo a radianes
  a = a * pi / 180.0

  ! ecuacion de la distancia horizontal
  d = u * u * sin(2 * a) / g

  ! escribiendo el resultado en la pantalla
  write(*,*) 'd: ',d

end program distancia_max
\end{verbatim}
\subsubsection{Ejemplo del programa}
Para $a$=$45$, y $u$=$9$ nos da una distancia $d$=$8.26530552$
\subsection{Conclusión}
Por conclusion, estos programas pueden ser de ayuda al momento de trabajar con el tiro parabolico y de ejemplo para otro tipo de programas con un proposito similar.
\end{document}
